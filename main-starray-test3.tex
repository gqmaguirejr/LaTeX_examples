\documentclass{article}
\RequirePackage[T1]{fontenc}
%\RequirePackage[utf8]{inputenc}
\usepackage{pgffor} % For \foreach loop and pgfkeys

\usepackage{starray} % For array definitions

% The following code is from Alceu Frigeri on 2025-06-10
% Modified by G. Q. Maguire to use ther S0, S1, S2, non-numeric index

% Writing it all in expl3
%
\ExplSyntaxOn


\msg_new:nnnn {st test} {unknown key}
  {
    Invalid~ key:~ '#1'
  }
  {
    Key~ not~ known: ~ '#1'
  }

\cs_generate_variant:Nn \msg_warning:nnn {nnx}

\keys_define:nn  { st test }
  {
    firstname .usage:n = general ,
    firstname .tl_set:N = \l_firstname_tl ,
    firstname .default:n = {} ,
    
    lastname .usage:n = general ,
    lastname .tl_set:N = \l_lastname_tl ,
    lastname .default:n = {} ,
    
    kthid .usage:n = general ,
    kthid .tl_set:N = \l_kthid_tl ,
    kthid .default:n = {} ,

    organization .usage:n = general ,
    organization .tl_set:N = \l_organization_tl ,
    organization .default:n = {} ,

    email .usage:n = general ,
    email .tl_set:N = \l_email_tl ,
    email .default:n = {} ,

    school .usage:n = general ,
    school .tl_set:N = \l_school_tl ,
    school .default:n = {} ,

    department .usage:n = general ,
    department .tl_set:N = \l_department_tl ,
    department .default:n = {} ,
       
    % not very efficient, but quick to write...
    default .usage:n = general ,
    default .meta:nn = {st test}
      { firstname , lastname, kthid, organization, email, school, department } ,
      
    unknown .code:n =
      {
        \msg_warning:nnx 
          {st test} {unknown key}
          {Unknown~ supervisor~ key:~ \l_keys_key_str}
      } ,
    unknown .default:V = \c_novalue_tl

  }




% 2. STARRAY SETUP: Declare the array to hold supervisor records (STANDARD LaTeX command)

\starray_new:n {SupervisorRecords}
\starray_def_from_keyval:nn {SupervisorRecords}
  {
    lastname = {} ,
    firstname = {} ,
    kthid = {} ,
    email = {} ,
    organization = {} ,
    school = {} ,
    department = {} ,    
  }
  % it can be done either way...
%\starray_def_prop:nnn {SupervisorRecords}{lastname}{}
%\starray_def_prop:nnn {SupervisorRecords}{firstname}{}
%\starray_def_prop:nnn {SupervisorRecords}{kthid}{}
%\starray_def_prop:nnn {SupervisorRecords}{email}{}
%\starray_def_prop:nnn {SupervisorRecords}{organization}{}
%\starray_def_prop:nnn {SupervisorRecords}{school}{}
%\starray_def_prop:nnn {SupervisorRecords}{department}{}



% use with care (see the use below)
\cs_generate_variant:Nn \starray_set_from_keyval:nn {ne}


% --- Helper Expl3 Macro to ADD SUPERVISOR DATA TO STARRAY ---
% This handles all expl3 and pgfkeys parsing for \addsupervisor
% #1 = A string of key=value pairs for one supervisor
\NewDocumentCommand{\addsupervisor}{m}
  {%

    \keys_set:nn { st test } {default , #1}

    \tl_if_empty:NTF \l_lastname_tl{
    Nothing~to~process\par
    } % end of true case
    { % start of false case
    start~of~false~case\par
    

    
    %% I would suggest the last name, for instance as a non numerical index
    %\starray_new_term:nn {SupervisorRecords}{\l_lastname_tl}
    % GQMJR: Alternatively, convert current length + 1 (an int) with prefix "S" to string: S1, S2, S3, ...)
    \starray_get_cnt:nN {SupervisorRecords} \l_tmpa_int     % Get the current count of records
    \int_incr:N \l_tmpa_int                                 % increment to be 1-base
    \tl_set:Nx \l_tmpa_tl { S\int_use:N \l_tmpa_int }       % prefix with "S"
    % Explicitly create a new term with this non-numerical ID.
    \starray_new_term:nn {SupervisorRecords}{\l_tmpa_tl} 
  
    %% note the full expantion. it is needed so the token list variables VALUE get used... use with care.
    \starray_set_from_keyval:ne {SupervisorRecords}
      {
        lastname = { \l_lastname_tl } ,
        firstname = { \l_firstname_tl } ,
        kthid = { \l_kthid_tl } ,
        email = { \l_email_tl } ,
        organization = { \l_organization_tl } ,
        school = { \l_school_tl } ,
        department = { \l_department_tl } ,    
      }
      %% OR JUST
      %% without using any explicit index it will always refers to the current iterator (which is the just created new term)

%    \starray_set_prop:nnV {SupervisorRecords}{lastname}    \l_lastname_tl  %% store
%    \starray_set_prop:nnV {SupervisorRecords}{firstname}   \l_firstname_tl  %% store
%    \starray_set_prop:nnV {SupervisorRecords}{kthid}       \l_kthid_tl  %% store
%    \starray_set_prop:nnV {SupervisorRecords}{email}       \l_email_tl  %% store
%    \starray_set_prop:nnV {SupervisorRecords}{organization} \l_organization_tl %% store
%    \starray_set_prop:nnV {SupervisorRecords}{school}      \l_school_tl %% store
%    \starray_set_prop:nnV {SupervisorRecords}{department}  \l_department_tl %% store
      

    } % end of false case 
    

}

\ExplSyntaxOff

\ExplSyntaxOn
\DeclareDocumentCommand \supervisorNames {}
{
\tl_new:N \g_supervisor_names_tl % Declare the global token list variable
\tl_new:N \l_lastnameI_tl         % Declare a local token list variable for the string to add
\tl_new:N \l_firstnameI_tl         % Declare a local token list variable for the string to add

% Initialize \g_supervisor_names_tl
\tl_set:Nn \l_supervisor_names_tl {}


\int_new:N \l_localcnt_int
\int_new:N \l_nextToLast_int
\int_new:N \l_aux_int
\int_set:Nn \l_aux_int { 1 }
\starray_get_cnt:nN {SupervisorRecords} \l_localcnt_int
\starray_get_cnt:nN {SupervisorRecords} \l_nextToLast_int
\int_incr:N \l_localcnt_int
\int_while_do:nNnn { \l_aux_int } < { \l_localcnt_int } {
    %     code using .... {SupervisorRecords[\l_aux_int]}
    %    like
      \starray_get_prop:nnN {SupervisorRecords[\int_use:N \l_aux_int]} {lastname} {\l_lastnameI_tl}
      \starray_get_prop:nnN {SupervisorRecords[\int_use:N \l_aux_int]} {firstname} {\l_firstnameI_tl}
      \int_compare:nNnTF {\l_aux_int} = {1}
        {\tl_set:Ne \l_supervisor_names_tl { \l_firstnameI_tl {~} \l_lastnameI_tl }}
        {\int_compare:nNnTF {\l_aux_int} = {\l_nextToLast_int}
            {\tl_put_right:Nn \l_supervisor_names_tl { ,~and~ }}
            {\tl_put_right:Nn \l_supervisor_names_tl { ,~ }}
            \tl_put_right:Ne \l_supervisor_names_tl { \l_firstnameI_tl {~} \l_lastnameI_tl }
        }
        {} % nothing more to do - there is only one author 
    \int_incr:N \l_aux_int
  }
\l_supervisor_names_tl
}
\ExplSyntaxOff


% --- DOCUMENT START ---
% GQMJR: cleaned-up the main document
\begin{document}
\section{Show basic array}
\ExplSyntaxOn
\starray_show_def_in_text:n {SupervisorRecords}\\
\par Terms:
\starray_show_terms_in_text:n {SupervisorRecords}\\
\ExplSyntaxOff

\section{Iterate through all supervisors (before filling)}
\ExplSyntaxOn

\starray_get_cnt:nN {SupervisorRecords} \l_tmpa_int
\starray_get_cnt:nN {SupervisorRecords} \l_tmpa_int
\int_compare:nNnTF {\l_tmpa_int} = {0}
  { No~supervisors}
  {
    \foreach \i in {1,...,\l_tmpa_int} {
        \starray_get_prop:nnN {SupervisorRecords[\i]} {lastname} {\lastname}
        Supervisor~{\i}:~\lastname\par
    }
  }
\par
\ExplSyntaxOff

\section{Fill-in the data}
% Populate Supervisor Data (using addsupervisor macro)
\addsupervisor{lastname={Maguire}, firstname={Gerald}, kthid={uld12345}, email={maguire@kth.se}, school={EECS}, department={Computer Science}}
\addsupervisor{lastname={Doe}, firstname={Jane}, email={jane.doe@example.com}, organization={University of Example}}
\addsupervisor{lastname={Smith}, firstname={John}, school={ITM}}
\addsupervisor{lastname={}} % Example of an empty supervisor (will be skipped in JSON output)

\section{Show basic array after filling in supervisors}
\ExplSyntaxOn
\starray_show_def_in_text:n {SupervisorRecords}\\
\par Terms:
\starray_show_terms_in_text:n {SupervisorRecords}\\
\ExplSyntaxOff

\section{Try getting some data about the supervisors}

\subsection{Get first supervisor's lastname }
\ExplSyntaxOn
\starray_get_prop:nnN {SupervisorRecords[1]} {lastname} {\lastname}\par
Get~lastname~of~first~supervisor:~\lastname
\ExplSyntaxOff

\section{Iterating over all terms}
\ExplSyntaxOn
\starray_iterate_over:nn {SupervisorRecords}
  {
    \noindent some~text ~ supervisor's ~ last ~name:~ \starray_get_prop:nn {SupervisorRecords} {lastname} ~and~ (if~ any)~ kthid:~ \starray_get_prop:nn {SupervisorRecords} {kthid} \par
  }
\ExplSyntaxOff

\subsection{Another way of iterating through all supervisors}
\ExplSyntaxOn
\starray_get_cnt:nN {SupervisorRecords} \l_tmpa_int
\int_compare:nNnTF {\l_tmpa_int} = {0}
  { No~supervisors}
  {
    \foreach \i in {1,...,\l_tmpa_int} {
        \starray_get_prop:nnN {SupervisorRecords[\i]} {lastname} {\lastname}
        Supervisor~{\i}:~\lastname\par
    }
  }
\ExplSyntaxOff

\section{An alternative iterator approach}
\supervisorNames


\section{Yet another iterator approach - output JSON-like output}

For the moment, we will just output the SupervisorRecords in a JSON-like format; later, they will be changed into writes to a file that can be attached to the PDF file.

\ExplSyntaxOn
\starray_reset_iter:n {SupervisorRecords}  % this will, in practice, set the current term index to 1
\bool_new:N \l_flag_bool
\bool_set_false:N \l_flag_bool
\bool_until_do:Nn \l_flag_bool 
 {
   % do something
    \starray_get_prop:nnN {SupervisorRecords} {lastname} \l_lastnameI_tl
    \starray_get_prop:nnN {SupervisorRecords} {firstname} \l_firstname_tl
    \starray_get_prop:nnN {SupervisorRecords} {kthid} \l_kthid_tl
    \starray_get_prop:nnN {SupervisorRecords} {email} \l_email_tl
    \starray_get_prop:nnN {SupervisorRecords} {organization} \l_organization_tl
    \starray_get_prop:nnN {SupervisorRecords} {school} \l_school_tl
    \starray_get_prop:nnN {SupervisorRecords} {department} \l_department_tl
    
    \starray_get_iter:nN {SupervisorRecords} \l_aux_int   % this will contain the index number of the current one
    \int_compare:nNnTF {\l_aux_int} = {1} {} {,\par}
    "Supervisor\int_use:N \l_aux_int”: \{ ”Last name”:~”\l_lastnameI_tl”
    ,~”First name”:~”\l_firstname_tl”
    \tl_if_empty:NTF \l_kthid_tl {} {,~”Local User Id”:~”\l_kthid_tl”}
    \tl_if_empty:NTF \l_email_tl {} {,~”E-mail”:~”\l_email_tl”}
    \tl_if_empty:NTF \l_school_tl {
        \tl_if_empty:NTF \l_organization_tl {} {,~”Other~organisation”:~”\l_organization_tl”}
    } {,~”organisation”: \{”L1”:~”\l_school_tl”}
         \tl_if_empty:NTF \l_department_tl {} {,~”L2”:~”\l_department_tl”}
        \}
    \}
 % ...

  % at the very end
    \starray_next_iter:nTF {SupervisorRecords}
    { } 
    { \bool_set_true:N \l_flag_bool } % this will end the loop.
  
 }
\ExplSyntaxOff


\end{document}
